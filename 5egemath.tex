\chapter{Задача 5}	
	\begin{enumerate}
		
	\item $ \log_{2}\left(4-x\right)=7 $
	\item $ \log_{5}\left(4+x\right)=2 $
	\item $ \log_{5}\left(5-x\right)=\log_{5}3 $
	\item $ \log_{2}\left(15+x\right)=\log_{2}3 $
	\item $ 2^{4-2x} = 64$
	\item $ 5^{x-7} = \cfrac{1}{125} $
	\item $ \left( \cfrac{1}{3}\right)^{x-8} = \cfrac{1}{9}$
	\item $ \left( \cfrac{1}{2}\right)^{6-2x} = 4$
	\item $ 16^{x-9} = \cfrac{1}{2} $
	\item $ \left( \cfrac{1}{9}\right) ^{x-13} = 3$
	\item $ \sqrt{15 - 2x} = 3  $
	\item $ \log_{4}\left(x+3\right) = \log_{4}\left(4x-15\right)$
	\item $ \log_{\frac{1}{7}}\left(7-x\right) = -2 $
	\item $ \log_{5}\left(5-x\right) = 2\log_{5}3 $
	\item $ \sqrt{\cfrac{6}{4x-54}} = \cfrac{1}{7} $
	\item $ \sqrt{\cfrac{2x+5}{3}} = 5 $
	\item $ \dfrac{4}{7}x = 7\dfrac{3}{7} $
	\item $ -\dfrac{2}{9}x = 1\dfrac{1}{9}$
	\item $ \dfrac{x-119}{x+7} = -5 $
	\item $ x = \dfrac{6x -15}{x-2}$ если уравнение имеет больше одного корня, в ответе запишите больший из корней.
	\item $ 9^{-5+x} = 729$ 
	\item Решите уравнение $ x^2 -17x + 72 = 0 $ если уравнение имеет больше одного корня, в ответе запишите меньший из корней.
	\item Решите уравнение $\sqrt{-72-17x} = -x$ если уравнение имеет больше одного корня, в ответе запишите меньший из корней.
	\item Решите уравнение $\cos{\dfrac{\pi(x-7)}{3}}  =  \dfrac{1}{2}$ если уравнение имеет больше одного корня, в ответе запишите больший из корней.
	\item $\left( \dfrac{1}{8}\right)^{-3+x} = 512 $
	\item $\left( \dfrac{1}{2}\right)^{x-8} = 2^x $
	\item $\sqrt[3]{x-4} = 3$
	\item $\dfrac{9}{x^2 - 16} = 1$ если уравнение имеет больше одного корня, в ответе запишите больший из корней.
	\item $\dfrac{13x}{2x^2 - 7} = 1$ если уравнение имеет больше одного корня, в ответе запишите меньший из корней.
	\item $\left(2x + 7\right) = \left(2x -1\right)  $
	\item $ \left( x-6\right)^2 = -24x $
	\item $ x^2 + 9 = \left(x+9\right)^2 $
	\item $ \dfrac{1}{3}x^2 = 16\dfrac{1}{3}$, если уравнение имеет больше одного корня, в ответе запишите меньший из корней.
	\item $ \left(x-1\right)^3 = -8 $
	\item $ \dfrac{1}{3x-4} = \dfrac{1}{4x-11}$
	\item $ \log_{8}2^{8x-4} = 4$
	\item $3^{\log_{9}\left(5x - 5\right) }  = 5$
	\item $\dfrac{x+8}{5x+7} = \dfrac{x+8}{7x+5}$ , если уравнение имеет больше одного корня, в ответе запишите больший из корней.
	\item $\sqrt{\dfrac{1}{15-4x}} = 0,2$
	\item $\sqrt{\dfrac{1}{5-2x}} = \dfrac{1}{3} $
	\item $\left( x-1\right)^3 = 8 $
	\item $\sqrt{6 + 5x} = x$, если уравнение имеет больше одного корня, в ответе запишите меньший из корней.
	\item $\tg{\dfrac{\pi x}{4}} = -1$. В ответ записать наибольший отрицательный корень.
	\item $\sin{\dfrac{\pi x}{3}} = 0,5$. В ответ записать наименьший положительный корень.
	\item $8^{9-x} = 64^x$
	\item $2^{3+x} = 0,4 \cdot 5^{3+x}$
	\item $ \log_{5}\left(x^2 + 2x \right) = \log_{5}\left( x^2 + 10\right)  $
	\item $ \log_{5}\left(7-x \right) = \log_{5}\left( 3-x\right) +1 $
	\item $ \log_{x-5}49 = 2 $, если уравнение имеет более одного корня, в ответе запишите меньший из корней.
	\item $\dfrac{1}{9x-7} = \dfrac{1}{2}$
	\item $ \dfrac{1}{4x-1} = 5$
	\item $ \sqrt{3x-8} = 5$

	\end{enumerate}