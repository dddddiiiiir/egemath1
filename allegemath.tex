\documentclass[a4paper]{book}
\usepackage[english,russian]{babel}
\usepackage[utf8x]{inputenc}
\usepackage{amsmath,amsthm,amssymb}
\newcommand{\RNumb}[1]{\uppercase\expandafter{\romannumeral #1\relax}}  % римские цифры становятся большими, какими и должны быть, чтобы записать, используется команда \RNumb{1}
\usepackage[margin=5cm]{geometry}
\usepackage{pgfplots}
\usepackage{tikz}
\pgfplotsset{width=7cm,compat=1.15}
\usepackage{mathrsfs}
\usetikzlibrary{arrows}
\usetikzlibrary{calc}
\pagestyle{empty}
\begin{document}
		
\tableofcontents
\backmatter

\chapter{Задача 1}	
	\begin{enumerate}
	
	\item  Шоколадка стоит 40 рублей. В воскресенье в супермаркете действует  
	специальное    предложение:  заплатив  за  две  шоколадки, покупатель  получает  три 
	(одну  –  в  подарок). Какое наибольшее  количество  шоколадок  можно  получить, 
	потратив не более 320 рублей в воскресенье? 
	
	\item Летом  килограмм  клубники  стоил  60  рублей. Маша  купила  2  кг  200  г  клубники. 
	Сколько рублей сдачи она должна была получить с 200 рублей?
	
	\item Показания счётчика электроэнергии 1 сентября составляли 79991 кВт$\cdot$ч, а 1 октября 
	–  80158  кВт$\cdot$ч.  Сколько  нужно  заплатить  за  электроэнергию  за  сентябрь, если  1  кВт$\cdot$ч 
	электроэнергии стоит 1 рубль 60 копеек? Ответ дайте в рублях. 
	
	\item Налог  на  доходы  составляет  13\%  от  заработной  платы.  Заработная  плата  Ивана 
	Кузьмича  равна  13000  рублей.  Какую  сумму  он  получит  после  вычета  налога  на 
	доходы? Ответ дайте в рублях. 
	
	\item Одна таблетка лекарства весит 40 мг и содержит 6\% активного вещества. Ребёнку в 
	возрасте  до  6  месяцев  врач  прописывает  1,2  мг  активного  вещества  на  каждый 
	килограмм  веса  в  сутки.  Сколько  таблеток  этого  лекарства  следует  дать  ребёнку  в 
	возрасте четырёх месяцев и весом 6 кг в течение суток? 
	
	\item По  результатам  приемной  кампании  2017  года,  в  вузы  на  бюджетные  места 
	поступили  6202  победителя  и  призера  олимпиад.  В  2016  году  этот  показатель 
	составлял 5950 человек. На сколько процентов был превышен показатель поступивших 
	в  вузы  на  бюджетные  места  победителей  и  призеров  олимпиад  в  2017  году  по 
	сравнению с 2016 годом? (Ответ округлите до целого числа процентов) 

	
	\item 27  выпускников  школы  поступили    в  технические  вузы.  Они  составляют  30\%  от 
	числа выпускников. Сколько в школе выпускников?
	
	\item На автозаправке клиент залил в бак 45 литров бензина по цене 36 руб. 60 коп. за 
	литр,  и  отдал  кассиру  пятитысячную  купюру.  Сколько  рублей  сдачи  он  должен 
	получить у кассира?
	
	\item Одна таблетка лекарства весит 20 мг и содержит 5\% активного вещества. Ребёнку в 
	возрасте  до  6  месяцев  врач  прописывает  1,4  мг  активного  вещества  на  каждый 
	килограмм  веса  в  сутки.  Сколько  таблеток  этого  лекарства  следует  дать  ребёнку  в 
	возрасте четырёх месяцев и весом 5 кг в течение суток? 
	
	\item Установка  двух  счётчиков  воды  (холодной  и  горячей)  стоит  3300  рублей.  До 
	установки  счётчиков  за  воду  платили  800  рублей  ежемесячно.  После  установки 
	счётчиков  ежемесячная  оплата  воды  стала  составлять  300  рублей.  Через  какое 
	наименьшее  количество  месяцев  экономия  по  оплате  воды  превысит  затраты  на 
	установку счётчиков, если тарифы на воду не изменятся? 
	
	\item Цена  на  электрический  чайник  была  повышена  на  14\%  и  составила  1596  рублей. 
	Сколько рублей стоил чайник до повышения цены?
	
	\item Света  отправила  SMS-сообщения  с  новогодними  поздравлениями  своим  19 
	друзьям. Стоимость одного SMS-сообщения  1  рубль  90  копеек.  Перед  отправкой  
	сообщения  на  счету  у  Светы  было  37  рублей. Сколько рублей останется у Светы 
	после отправки всех сообщений? 
	
	\item Установка  двух  счётчиков  воды  (холодной  и  горячей)  стоит  2000  рублей.  До 
	установки  счётчиков  за  воду  платили  1500  рублей  ежемесячно.  После  установки 
	счётчиков  ежемесячная  оплата  воды  стала  составлять  1200  рублей.  Через  какое 
	наименьшее  количество  месяцев  экономия  по  оплате  воды  превысит  затраты  на 
	установку счётчиков, если тарифы на воду не изменятся?
	
	\item Для  приготовления  яблочного  варенья  на  1  кг  яблок  нужно  1,2  кг  сахара.  Какое 
	наименьшее  количество  килограммовых  упаковок  сахара  нужно,  чтобы  сварить 
	варенье из 14 кг яблок?
	
	\item При    оплате    услуг    через    платежный    терминал    взимается    комиссия    8\%.  
	Терминал  принимает  суммы, кратные 10 рублям. Аня хочет положить на счет своего 
	мобильного телефона не меньше 500 рублей. Какую минимальную сумму она должна 
	положить в приемное устройство данного терминала? 
	
	\item Рост  Майкла  5  футов  3  дюйма.  Выразите  рост Майкла  в  сантиметрах,  если  1  фут 
	равен 0,305 м, а 1 дюйм равен 2,54 см. Результат округлите до целого числа. 
	
	\item  Пётр Иванович  купил  американский автомобиль,  спидометр  которого  показывает 
	скорость  в  милях  в  час.  Какова  скорость  автомобиля  в  километрах  в  час,  если 
	спидометр  показывает  28  миль  в  час?  Считайте,  что  1  миля  равна  1609  м.  Ответ 
	округлите до целого числа. 
	
	\item На контрольной работе по математике 60\% учеников писали первый вариант, треть 
	учеников  класса  писали  второй  вариант,  а  двое  не  писали  контрольную  (Саша  –  по 
	болезни, а Маша проспала). Сколько учеников в классе?
	
	\item Аня отправила SMS-сообщения с новогодними поздравлениями своим 19 друзьям. 
	Стоимость одного SMS-сообщения 1 рубль 90 копеек. Перед отправкой сообщения на 
	счету  у  Ани  было  37  рублей.  Сколько  рублей  останется  у  Ани  после  отправки  всех 
	сообщений?
	
	\item Из  одного  листа  бумаги  при  печати  получается  4  книжные  страницы. Сколько  
	пачек  бумаги  (по  500  листов  в  каждой)  необходимо,  чтобы  издать книгу тиражом 
	3000 экземпляров, в которой 55 страниц? 
	
	\item  Когда  Аристарх  Луков-Арбалетов  сдал  ОГЭ,  друзья  подарили  ему  10  биткоинов. 
	Сколько  раз  Аристарх  может  оплатить  6-летннее  обучение  в  ВУЗе,  если  стоимость 
	обучения 300 тыс. рублей за год, к моменту оплаты курс биткоина был 17000 долларов 
	США, а один доллар стоил 57 рублей?
	
	\item Тетрадь стоит 24 рубля. Сколько рублей заплатил покупатель за 60 тетрадей, если 
	при  покупке  больше  50  тетрадей  магазин  делает  скидку  10\%  от  стоимости  всей 
	покупки?
	 
	\item Пирожок в кулинарии стоит 12 рублей. При покупке более 30 пирожков продавец 
	делает скидку 5\% от стоимости всей покупки. Покупатель купил 40 пирожков. Сколько 
	рублей он заплатил за покупку? 
	
	\item В сентябре 1 кг винограда стоил 60 рублей, в октябре виноград подорожал на 25%, 
	а  в  ноябре  еще  на  20\%.  Сколько  рублей  стоил  1  кг  винограда  после  подорожания  в 
	ноябре?
	
	\item Установка  двух  счётчиков  воды  (холодной  и  горячей)  стоит  2000  рублей.  До 
	установки  счётчиков  за  воду  платили  1500  рублей  ежемесячно.  После  установки 
	счётчиков  ежемесячная  оплата  воды  стала  составлять  1200  рублей.  Через  какое 
	наименьшее  количество  месяцев  экономия  по  оплате  воды  превысит  затраты  на 
	установку счётчиков, если тарифы на воду не изменятся? 
	
	\item В доме, в котором живет Петя, один подъезд. На каждом этаже (включая первый) 
	по шесть квартир. Петя живет в квартире №50. На каком этаже живет Петя?
	
	\item В школе №1 уроки начинаются в 8:30, каждый урок длится 45 минут, все перемены, 
	кроме  одной,  длятся  10  минут,  а  перемена  между  вторым  и  третьим  уроком—20 
	минут.  Сейчас  на  часах  13:00.  Через  сколько  минут  прозвенит  ближайший  звонок  с 
	урока? 
	
	\item Большой  корабль  не  может  подойти  к  берегу,  поэтому  пассажиров  отвозят  с 
	корабля  на  шлюпке,  вмещающей  8  пассажиров.  Сколько  раз  шлюпка  приставала  к 
	берегу, если на берег отвезли 30 пассажиров?
	
	\item Оптовая  цена  учебника  140  рублей.  Розничная  цена  на  50\% выше  оптовой.  Какое 
	наибольшее число таких учебников можно купить по розничной цене на 5000 рублей?
	
	\item В сентябре 1 кг винограда стоил 60 рублей, в октябре виноград подорожал на 25\%, 
	а  в  ноябре  еще  на  20\%.  Сколько  рублей  стоил  1  кг  винограда  после  подорожания  в 
	ноябре?
			
	\item Тетрадь стоит 40 рублей. Какое наибольшее количество тетрадей можно купить на 750 рублей после понижения цены на 10\%?
	
	\item Магазин закупает цветочные горшки по оптовой цене 120 рублей за штуку и продает с наценкой 20\%. Какое наибольшее количество таких горшков можно купить в этом магазине на 1000 рублей?
	
	\item В пачке 500 листов бумаги формата А4. За неделю в офисе расходуется 1200 листов. Какого наименьшего количества пачек бумаги хватит на 4 недели?
	
	\item Стоимость проездного билета на месяц составляет 580 рублей, а стоимость билета на одну поездку - 20 рублей. Аня купила проездной и сделала за месяц 41 поездку. На сколько рублей больше она бы потратила, еслиorder бы покупала билеты на одну поездку?
	
	\item Больному прописано лекарство, которое нужно принимать по 0,4 г 3 раза в день в течение 21 дня. В одной упаковке 10 таблеток лекарства по 0,5 г. Какого наименьшего количества упаковок хватит на весь курс лечения?
	
	\item Для приготовления маринада для огурцов на 1 литр воды требуется 12 г лимонной кислоты. Лимонная кислота продается в пакетиках по 10 г. Какое наименьшее число пачек нужно купить хозяйке для приготовления 6 литров маринада?
	
	\item Шоколадка стоит 34 рублей. В воскресенье в супермаркете действует специальное предложение: заплатив за две шоколадки, покупатель получает три(одну - в подарок). Какое наибольшее количество шоколадок можно получить, потратив не более 200 рублей?
	
	\item Оптовая цена учебника 170 рублей. Розничная цена на 20\% выше оптовой. Какое наибольшее число таких учебников можно купить по розничной цене на 7000 рублей?
	
	\item Железнодорожный билет для взрослого стоит 720 рублей. Стоимость билета для школьника составляет 50\% от стоимости билета для взрослого. Группа состоит из 15 школьников и 2 взрослых. Сколько рублей стоят билеты на всю группу?
	
	\item Цена на электрический чайник была повышена на 16\% и составила 3480 рублей. Сколько рублей стоил чайник до повышения цены?

	
	
\end{enumerate}


\chapter{Задача 2}	
\begin{enumerate}
		
	\item Тетрадь стоит 40 рублей. Какое наибольшее количчество тетрадей можно купить на 750 рублей после понижения цены на 10\%?
	
\end{enumerate}


\chapter{Задача 3}	
\begin{enumerate}
		
	\item Тетрадь стоит 40 рублей. Какое наибольшее количчество тетрадей можно купить на 750 рублей после понижения цены на 10\%?
	
\end{enumerate}


\chapter{Задача 4}	
	\begin{enumerate}
		
	\item Тетрадь стоит 40 рублей. Какое наибольшее количчество тетрадей можно купить на 750 рублей после понижения цены на 10\%?
	
	\item Прямая $ y = -4x - 11 $ является касательной к графику функции $ y = x^3 + 7x^2 + 7x - 6 $. Найдите абсциссу точки касания.
			
	\item Прямая $ y = 7x - 5 $ параллельна касательной к графику функции $ y = x^2 + 6x - 8$. Найдите абсциссу точки касания.
	
	\item На рисунке изображен график функции $ y = f(x) $, определенной на интервале (-3; 9). Найдите количество решений уравнения $ f'(x) = 0 $ на отрезке [0; 8].
	
\end{enumerate}


\chapter{Задача 5}	
	\begin{enumerate}
		
	\item $ \log_{2}\left(4-x\right)=7 $
	\item $ \log_{5}\left(4+x\right)=2 $
	\item $ \log_{5}\left(5-x\right)=\log_{5}3 $
	\item $ \log_{2}\left(15+x\right)=\log_{2}3 $
	\item $ 2^{4-2x} = 64$
	\item $ 5^{x-7} = \cfrac{1}{125} $
	\item $ \left( \cfrac{1}{3}\right)^{x-8} = \cfrac{1}{9}$
	\item $ \left( \cfrac{1}{2}\right)^{6-2x} = 4$
	\item $ 16^{x-9} = \cfrac{1}{2} $
	\item $ \left( \cfrac{1}{9}\right) ^{x-13} = 3$
	\item $ \sqrt{15 - 2x} = 3  $
	\item $ \log_{4}\left(x+3\right) = \log_{4}\left(4x-15\right)$
	\item $ \log_{\frac{1}{7}}\left(7-x\right) = -2 $
	\item $ \log_{5}\left(5-x\right) = 2\log_{5}3 $
	\item $ \sqrt{\cfrac{6}{4x-54}} = \cfrac{1}{7} $
	\item $ \sqrt{\cfrac{2x+5}{3}} = 5 $
	\item $ \dfrac{4}{7}x = 7\dfrac{3}{7} $
	\item $ -\dfrac{2}{9}x = 1\dfrac{1}{9}$
	\item $ \dfrac{x-119}{x+7} = -5 $
	\item $ x = \dfrac{6x -15}{x-2}$ если уравнение имеет больше одного корня, в ответе запишите больший из корней.
	\item $ 9^{-5+x} = 729$ 
	\item Решите уравнение $ x^2 -17x + 72 = 0 $ если уравнение имеет больше одного корня, в ответе запишите меньший из корней.
	\item Решите уравнение $\sqrt{-72-17x} = -x$ если уравнение имеет больше одного корня, в ответе запишите меньший из корней.
	\item Решите уравнение $\cos{\dfrac{\pi(x-7)}{3}}  =  \dfrac{1}{2}$ если уравнение имеет больше одного корня, в ответе запишите больший из корней.
	\item $\left( \dfrac{1}{8}\right)^{-3+x} = 512 $
	\item $\left( \dfrac{1}{2}\right)^{x-8} = 2^x $
	\item $\sqrt[3]{x-4} = 3$
	\item $\dfrac{9}{x^2 - 16} = 1$ если уравнение имеет больше одного корня, в ответе запишите больший из корней.
	\item $\dfrac{13x}{2x^2 - 7} = 1$ если уравнение имеет больше одного корня, в ответе запишите меньший из корней.
	\item $\left(2x + 7\right) = \left(2x -1\right)  $
	\item $ \left( x-6\right)^2 = -24x $
	\item $ x^2 + 9 = \left(x+9\right)^2 $
	\item $ \dfrac{1}{3}x^2 = 16\dfrac{1}{3}$, если уравнение имеет больше одного корня, в ответе запишите меньший из корней.
	\item $ \left(x-1\right)^3 = -8 $
	\item $ \dfrac{1}{3x-4} = \dfrac{1}{4x-11}$
	\item $ \log_{8}2^{8x-4} = 4$
	\item $3^{\log_{9}\left(5x - 5\right) }  = 5$
	\item $\dfrac{x+8}{5x+7} = \dfrac{x+8}{7x+5}$ , если уравнение имеет больше одного корня, в ответе запишите больший из корней.
	\item $\sqrt{\dfrac{1}{15-4x}} = 0,2$
	\item $\sqrt{\dfrac{1}{5-2x}} = \dfrac{1}{3} $
	\item $\left( x-1\right)^3 = 8 $
	\item $\sqrt{6 + 5x} = x$, если уравнение имеет больше одного корня, в ответе запишите меньший из корней.
	\item $\tg{\dfrac{\pi x}{4}} = -1$. В ответ записать наибольший отрицательный корень.
	\item $\sin{\dfrac{\pi x}{3}} = 0,5$. В ответ записать наименьший положительный корень.
	\item $8^{9-x} = 64^x$
	\item $2^{3+x} = 0,4 \cdot 5^{3+x}$
	\item $ \log_{5}\left(x^2 + 2x \right) = \log_{5}\left( x^2 + 10\right)  $
	\item $ \log_{5}\left(7-x \right) = \log_{5}\left( 3-x\right) +1 $
	\item $ \log_{x-5}49 = 2 $, если уравнение имеет более одного корня, в ответе запишите меньший из корней.
	\item $\dfrac{1}{9x-7} = \dfrac{1}{2}$
	\item $ \dfrac{1}{4x-1} = 5$
	\item $ \sqrt{3x-8} = 5$

	\end{enumerate}

\chapter{Задача 6}	
\begin{enumerate}
		
	\item Тетрадь стоит 40 рублей. Какое наибольшее количчество тетрадей можно купить на 750 рублей после понижения цены на 10\%?
	
\end{enumerate}


\chapter{Задача 7}	


	\begin{enumerate}
	
	
	
	\item Прямая $ y = -4x - 11 $ является касательной к графику функции $ y = x^3 + 7x^2 + 7x - 6 $. Найдите абсциссу точки касания.
	
	
	
	\item Прямая $ y = 7x - 5 $ параллельна касательной к графику функции $ y = x^2 + 6x - 8$. Найдите абсциссу точки касания.
	
	
	
	\item На рисунке изображен график функции $ y = f(x) $, определенной на интервале (-12; 8). Найдите количество решений уравнения $ f'(x) = 0 $ на отрезке [-8; 5].
	
	\resizebox{12cm}{6cm}
	{
		\begin{tikzpicture}
		\begin{axis}[
		enlarge x limits=false,
		smooth,
		enlarge x limits=false,
		xshift=1,
		xmin=-11,xmax=11,
		ymin=-6,ymax=6,
		grid=both,
		axis lines=middle,
		minor tick num=4,
		enlargelimits={abs=1},
		axis line style={latex-latex},
		ticklabel style={font=\tiny,fill=white},
		xlabel style={at={(ticklabel* cs:1)},anchor=north west},
		ylabel style={at={(ticklabel* cs:1)},anchor=south west}
		]
		
		\coordinate (O) at (0,0);
		\node[fill=white,circle,inner sep=0pt] (O-label) at ($(O)+(-135:10pt)$) {$O$};
		
		\addplot coordinates 
		{(-12,3) (-9,-3) (-7,4) (-3,-4) (-1,5) (2,2) (6,5) (8,-1)};
		
		\\closedcycle;
		\end{axis}
		\end{tikzpicture}
		
	}
	
	\end{enumerate}

\chapter{Задача 8}	
	\begin{enumerate}
		
	\item Тетрадь стоит 40 рублей. Какое наибольшее количчество тетрадей можно купить на 750 рублей после понижения цены на 10\%?
	
\end{enumerate}


\chapter{Задача 9}	
	\begin{enumerate}
	\item $\sqrt{65^2 - 56^2}$
	\item $\cfrac{\left( 2\sqrt{7}\right)^2 }{14}$
	\item $\left(\sqrt{13} - \sqrt{7} \right) \left( \sqrt{13} + \sqrt{7}\right) $
	\item $5^{0,36} \cdot 25^{0,32}$
	\item $\dfrac{3^{6,5}}{9^{2,25}}$
	
	\item $7^{\frac{4}{9}} \cdot 49^{\frac{5}{18}}$
	\item $\cfrac{2^{3,5} \cdot 3^{5,5}}{6^{4,5}}$
	\item $35^{-4,7} \cdot 7^{5,7} : 5^{-3,7}$
	\item $\cfrac{\sqrt{2,8} \cdot \sqrt{4,2}}{\sqrt{0,24}}$
	\item $\left( \sqrt{3\frac{6}{7}} - \sqrt{1\frac{5}{7}}\right) : \sqrt{\frac{3}{28}}$ 
		
	\item $\frac{\sqrt[9]{7} \cdot \sqrt[18]{7}}{\sqrt[6]{7}}$	
	\item $\frac{\sqrt[5]{10} \cdot \sqrt[5]{16}}{\sqrt[5]{5}}$
	\item $\left( \dfrac{2^{\frac{1}{3}} \cdot 2^{\frac{1}{4}}}{\sqrt[12]{2}}\right)^2 $
	\item $\cfrac{\left( 2^{\frac{3}{5}} \cdot 5^{\frac{2}{3}}\right)^{15} }{10^9}$	
	\item $0,8^{\frac{1}{7}} \cdot 5^{\frac{2}{7}} \cdot 20^{\frac{6}{7}}$
	
	\item $\cfrac{\left(\sqrt{13} + \sqrt{7} \right)^2}{10 +\sqrt{91}}$
	\item $5 \cdot \sqrt[3]{9} \cdot \sqrt[6]{9}$
	\item $\cfrac{49^{5,2}}{7^{8,4}}$
	\item $\cfrac{12 \sin{11^{\circ}} \cos{11^{\circ}}}{\sin{22^{\circ}}}$
	\item $\cfrac{24\left( \sin^{2}{17^{\circ}} - \cos^{2}{17^{\circ}}\right) }{\cos{34^{\circ}}}$
	
	\item $\cfrac{5\cos{29^{\circ}}}{\sin{61^{\circ}}}$
	\item $36\sqrt{6}\tg{\frac{\pi}{6}}\sin{\frac{\pi}{4}}$
	\item $4\sqrt{2}\cos{\frac{\pi}{4}}\sin{\frac{7\pi}{3}}$
	\item $\dfrac{8}{\sin{\left( -\frac{27\pi}{4}\right) }\cos{\left( \frac{31\pi}{4}\right) }}$
	\item $-4\sqrt{3} \cos{\left( -750^{\circ}\right) }$
	
	\item $2\sqrt{3}\tg{\left( -300^{\circ}\right) }$
	\item $-18\sqrt{2}\sin{\left( -135^{\circ}\right) }$
	\item $24\sqrt{2}\cos{\left( -\frac{\pi}{3}\right) } \sin{\left( -\frac{\pi}{4}\right) }$
	\item $\dfrac{14\sin{19^{\circ}}}{\sin{341^{\circ}}}$
	\item $\dfrac{4\cos{146^{\circ}}}{\cos{34^{\circ}}}$
	
	\item $\dfrac{5\tg{163^{\circ}}}{\tg{17^{\circ}}}$
	\item $5\tg{17^{\circ}} \cdot \tg{107^{\circ}}$
	\item $7\tg{13^{\circ}} \cdot \tg{77^{\circ}}$
	\item $\dfrac{12}{\sin^{2}{37^{\circ}} + \sin^{2}{127^{\circ}}}$ 
	\item $\dfrac{6}{\cos^{2}{23^{\circ}} + \cos^{2}{113^{\circ}}}$
	 
	\item $\dfrac{12}{\sin^{2}{27^{\circ}} + \cos^{2}{207^{\circ}}}$ 
	\item найдите $\tg{\alpha}$, если $\cos{\alpha} = \frac{\sqrt{10}}{10}$ и $\alpha \in \left(\frac{3\pi}{2}; 2\pi\right) $
	\item найдите $\tg{\alpha}$, если $\sin{\alpha} = -\frac{5}{\sqrt{26}}$ и $\alpha \in \left(\pi; \frac{3\pi}{2}\right) $
	\item найдите $3\cos{\alpha}$, если $\sin{\alpha} = -\frac{2\sqrt{2}}{3}$ и $\alpha \in \left(\frac{3\pi}{2}; 2\pi\right) $
	\item найдите $5\sin{\alpha}$, если $\cos{\alpha} = \frac{2\sqrt{6}}{5}$ и $\alpha \in \left(\frac{3\pi}{2}; 2\pi\right) $
	
	\item найдите $24\cos{2\alpha}$, если $\sin{\alpha} = -0,2$
	\item найдите $\frac{10\sin{6\alpha}}{3\cos{3\alpha}}$, если $\sin{3\alpha} = 0,6$
	\item
	\item
	\item
	
	\item
	\item
	\item
	\item
	\item
	
	\item
	\item
	\item
	\item
	\item
	
	\item
	\item
	\item
	\item
	\item
	
	\item
	\item
	\item
	\item
	\item
	
	\item
	\item
	\item
	\item
	\item
	
	\item
	\item
	\item
	\item
	\item
	
	\item
	\item
	\item
	\item
	\item
	
	\item
	\item
	\item
	\item
	\item
	
	\item
	\item
	\item
	\item
	\item
	
	\item
	\item
	\item
	\item
	\item
	
	\item
	\item
	\item
	\item
	\item
	
	\item
	\item
	\item
	\item
	\item
	
	\item
	\item
	\item
	\item
	\item
	
	\item
	\item
	\item
	\item
	\item
	
	\item
	\item
	\item
	\item
	\item
	
	\item
	\item
	\item
	\item
	\item
	
	\item
	\item
	\item
	\item
	\item
	
	\item
	\item
	\item
	\item
	\item
	
	\item
	\item
	\item
	\item
	\item
	
	\item
	\item
	\item
	\item
	\item
	
	\item
	\item
	\item
	\item
	\item
	
	\item
	\item
	\item
	\item
	\item
	
	\item
	\item
	\item
	\item
	\item
	
	\item
	\item
	\item
	\item
	\item
	
	\item
	\item
	\item
	\item
	\item
	
	\item
	\item
	\item
	\item
	\item
	
	\item
	\item
	\item
	\item
	\item
	
	\item
	\item
	\item
	\item
	\item
	
	\item
	\item
	\item
	\item
	\item
	
	\item
	\item
	\item
	\item
	\item
	
	\item
	\item
	\item
	\item
	\item
	
	
	
	
	
	
	
	
	
	
	
	$90^{\circ}$

\end{enumerate}

\chapter{Задача 10}	
\begin{enumerate}
		
	\item Тетрадь стоит 40 рублей. Какое наибольшее количчество тетрадей можно купить на 750 рублей после понижения цены на 10\%?
	
\end{enumerate}


\chapter{Задача 11}	
\begin{enumerate}
		
	\item Тетрадь стоит 40 рублей. Какое наибольшее количчество тетрадей можно купить на 750 рублей после понижения цены на 10\%?
	
\end{enumerate}

\chapter{Задача 12}	
\begin{enumerate}
		
	\item Тетрадь стоит 40 рублей. Какое наибольшее количчество тетрадей можно купить на 750 рублей после понижения цены на 10\%?
	
\end{enumerate}

\chapter{Задача 13}	
	
\begin{enumerate}
		
	\item Тетрадь стоит 40 рублей. Какое наибольшее количчество тетрадей можно купить на 750 рублей после понижения цены на 10\%?
	
\end{enumerate}


\chapter{Задача 14}	
\begin{enumerate}
		
	\item Тетрадь стоит 40 рублей. Какое наибольшее количчество тетрадей можно купить на 750 рублей после понижения цены на 10\%?
	
\end{enumerate}

\chapter{Задача 15}	
\begin{enumerate}
		
	\item Тетрадь стоит 40 рублей. Какое наибольшее количчество тетрадей можно купить на 750 рублей после понижения цены на 10\%?
	
\end{enumerate}

\chapter{Задача 16}	
	\begin{enumerate}
		
	\item Тетрадь стоит 40 рублей. Какое наибольшее количчество тетрадей можно купить на 750 рублей после понижения цены на 10\%?
	
\end{enumerate}


\chapter{Задача 17}	
\begin{enumerate}
		
	\item Тетрадь стоит 40 рублей. Какое наибольшее количчество тетрадей можно купить на 750 рублей после понижения цены на 10\%?
	
\end{enumerate}

\chapter{Задача 18}	
\begin{enumerate}
		
	\item Тетрадь стоит 40 рублей. Какое наибольшее количчество тетрадей можно купить на 750 рублей после понижения цены на 10\%?
	
\end{enumerate}


\chapter{Задача 19}	
	\begin{enumerate}
		
	\item Тетрадь стоит 40 рублей. Какое наибольшее количчество тетрадей можно купить на 750 рублей после понижения цены на 10\%?
	
\end{enumerate}


\end{document}